LOAT.0.00.00.00.000
11717.0.01.01.00.000
11588.0.01.01.00.000
11706.0.01.01.00.000
11825.0.00.01.00.000
11579.0.01.04.00.000
11826.0.00.01.00.000
I0933.0.15.02.04.018




The last problem above is more complicated, but can be done in one line using one of the higher-order list functions you've been working with. 




%%%%%%Legacy

<!---        

<div class="comment">JK: Comment from Dan and NY teachers: "They worked for quite a while, not realizing that if the function passed to map is �item _ of LIST� that they could remove the 1 to leave a blank slot.  This has come up before with students who see a drop-down menu and don�t realize it can be edited."</div>
        
        <li><strong>Make</strong> a red block, <img border="0" src="/bjc-r/img/3-lists/lookup_block.png" alt="red reporter: lookup items () from ()" title="red reporter: lookup items () from ()" />. Here's how it starts:<br> <img border="0" src="/bjc-r/img/3-lists/lookup_block_def.png" alt="lookup items (item numbers) from (this list)" title="lookup items (item numbers) from (this list)" /><br>And here it is in action on our list <code>nouns</code>:<div class="sidenote">If you need some inspiration, look again at <a href="/bjc-r/cur/programming/3-lists/2-list-operations/getting-started.html#fmap">problem 3 in the Getting Started</a> for this lab.</div><br><img border="0" src="/bjc-r/img/3-lists/nouns.png" alt="giraffe, elephant, pizza, etc." title="giraffe, elephant, pizza, etc." /> <br> <img border="0" src="/bjc-r/img/3-lists/lookup_result.png" alt="lookup items (1,2,5) from (nouns) = (giraffe, elephant, boy)" title="lookup items (1,2,5) from (nouns) = (giraffe, elephant, boy)" /></li>
        

 
        
        
        
        
        
        
        
        
                
        <li>Once you've made <code>lookup items () from ()</code>, go to the file menu <img class="button" src="/bjc-r/img/1-introduction/file_button.png" alt="File button"/> and choose "Export blocks...". This opens a window with all the imported tools and all your custom blocks. <strong>Uncheck</strong> all the blocks <strong>except</strong> your new <code>lookup</code> block. Choose "ok."<br> You will see a new tab open with text. Save this as "lookup.xml" (go to your browser's File->Save Page As... or right-click on the page and Save As...). Save it to a folder or on your computer's desktop.<br>Now, you can use this block in other <span class="snap">snap</span> projects. Try it now. Open a new <span class="snap">snap</span> page. Then <strong>drag</strong> the "lookup.xml" file onto the <span class="snap">snap</span> page. The block should now be in your Variables palette.<br>You will use the <code>lookup</code> block again later in Unit 3 Lab 4.</li>
        
    </ol>
    </div>
--->